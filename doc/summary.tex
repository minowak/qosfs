\chapter{Podsumowanie i wnioski} \label{ch:summary}

\section{Wnioski}

Postawione cele zostały osiągnięte. Zaimplementowano system plików z mechanizmem
QoS, który pozwala na kontrolowanie wykorzystania przepustowości dla
operacji I/O. Pozwala on na ustalenie minimalnej szybkosci, która w miarę możliwości
nie zostaje przekroczona.

Pomimo skoków szybkości transferu danych ponad narzucony limit,
utrzymuje się ona w okolicach wyznaczonych wartości. Skoki te są
tylko chwilowe i niewielkie (najwyższy rzędu 11\% wartości limitu i 3,6\% przepustowości dysku, nie biorąc pod uwagę
skoków w 1 sekundzie) i nie zakłócają działania systemu.

Przeanalizowano system pod kątem szybkości transferu, ilości operacji na sekundę
oraz opóźnienia dysku. QoSFS działa, i może być przydatny w sytuacjach wymagających
stabilizacji szybkości transferu dla wybranych katalogów lub plików.

\section{Możliwości rozwoju}
Poniżej przedstawiono kilka przykładowych możliwości rozwoju projektu:
\ \\
\begin{enumerate}
	\item \textbf{Przechowywanie limitów szybkości transferu danych w inodach plików}. Dzięki takiemu zabiegowi możliwe będzie 
    nadanie osobnych limitów dla każdego pliku, a umieszczenie ich w inodach pozwoli na ich łatwe
     ustalanie z poziomu użytkownika. FUSE pozwala na tworzenie własnych implementacji
    inodów.
    
    \item \textbf{Zmienny deadline}. Deadline liczony dla każdego pliku w czasie wykonywania
    operacji I/O. Czas ten może zależeć np. od wielkości lub priorytetu pliku.
    
    \item \textbf{Dodatkowe schedulery}. Implementacja dodatkowych schedulerów biorących pod uwagę
    inne właściwości pliku, na którym wykonywana jest operacja.
    
    \item \textbf{Możliwośc ustalenia wykorzystywanej przepustowości}. Na chwilę obecną dostępna przepustowość
    liczona jest na podstawie różnicy pomiędzy maksymalną możliwą, a aktualnie używaną. 
    Nadanie tego limitu umożliwi używanie tylko części przepustowości
    dysku zapewniając zawsze dostępną minimalną szybkość transferu danych dla operacji na innych systemach plików.
    
    \item \textbf{Możliwośc przekroczenia ustalonej przepustowości}. Pozwolenie
    operacjom na przekroczenie narzuconej przepustowości w określonych
    warunkach jeżeli większa jest dostepna.
\end{enumerate}