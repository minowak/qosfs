\chapter{Rozważane rozwiązania} \label{ch:prototypes}

\section{Sterowanie przepustowością dysku}\label{throttlingfailures}

Pierwszym problem, jaki należało rozwiązać, był wybór sposobu sterowania
przepustowością dysku. Na samym początku rozważane było napisanie modułu do kernela, który
pozwoliłby na bezpośrednie kontrolowanie szybkości transferu danych, ale przewidziane problemy z współbieżnością
oraz trudności w instalacji i testowaniu spowodowały porzucenie tego rozwiązania.

\subsection{hdparm}
Pierwszym narzędziem do kontroli przepustowości dysku, jaki testowano było \texttt{hdparm}
\footnote{https://sourceforge.net/projects/hdparm/}. Narzędzie zostało napisane w 2005 roku przez
kanadyjczyka Marka Lorda. Używane jest do diagnostyki, testowania oraz dostrajania parametrów dysków.

Warto zaznaczyć, że nierozważne używanie hdparm może spowodować utratę danych i, w najgorszym
przypadku, fizyczny defekt urządzenia.

Możliwośc uszkodzenia dysku oraz krótkie testy funkcjonalności stwierdziły, że hdparm nie nadaje się do celów tego projektu.

\subsection{ionice}
Odpowiednikiem linuxowego narzędzia \texttt{nice}\footnote{http://linux.die.net/man/1/nice}
dla pamieci dyskowych jest \texttt{ionice}\footnote{http://linux.die.net/man/1/ionice}.

Pomimo, iż posiada ono możliwość nadawania priorytetów oraz kontrolowania nimi przepustowości
nie ma możliwości narzucenia twardych limitów dla szybkości transferu danych, która jest 
decydująca do osiągnięcia celów niniejszego projektu.

\subsection{cgroups}
Grupy kontrolne \texttt{cgroups}\footnote{https://www.kernel.org/doc/Documentation/cgroup-v1/cgroups.txt} zapewniają możliwość grupowania procesów oraz nadają im limity wykorzystywania zasobów
(CPU, pamięć oraz I/O) oraz narzucania im twardych granic. 

Narzędzia cgroups są łatwo dostępne dla większości znanych dystybucji
systemu Linux oraz posiadają bogatą dokumentację oraz poradniki od \textit{Red Hata}
\footnote{https://access.redhat.com/documentation/en-US/Red\_Hat\_Enterprise\_Linux/6/html/Resource\_Management\_Guide/ch01.html}
i \textit{Arch Linuxa}\footnote{https://wiki.archlinux.org/index.php/Cgroups}. 
Dodatkowo powstała
również biblioteka - \texttt{libcgroups}, która pozwala na używanie grup
kontrolnych z poziomu języka C.

Grupami można zarządzać używając dostarczonych narzędzi \ \texttt{cgcreate}, \texttt{cgset}, 
\texttt{cgexec}, \texttt{cgremove} albo operując bezpośrednio na plikach konfiguracyjnych.

Grupy kontrolne zostały porzucone na rzecz własnej implementacji kontroli przepustowości, ponieważ
działają one tylko na procesach operujących na fizycznych nośnikach danych i nie działają
na implementacjach systemów plików opartych na VFS.

\section{Linux I/O scheduler}
W celu kolejkowania operacji I/O rozważane było użycie jednego z dostępnych schedulerów
Linuxa\footnote{https://www.kernel.org/doc/Documentation/scheduler/}, a w szczególności
\texttt{sched-deadline}.

Rozwiązania te jednak nie są możliwe do skonfigurowania w taki sposób, aby możliwe było nadawanie
różnych priorytetów operacjom w zależności od typu przetwarzanego pliku. Dlatego postanowiono
stworzyć własny scheduler, operujący w przestrzeni użytkownika.

\section{Extended attributes}\label{sec:xattrs}
Początkowo rozpatrywana była możliwośc nadawania limitów szybkości transferu danych dla każdego pliku z osobna.
Do tego celu wykorzystywane były rozszerzone atrybuty - \texttt{xattrs}\footnote{http://linux.die.net/man/2/getxattr}. W chwili obecnej limity ustalane 
są podczas montowania systemu plików, dlatego porzucono to rozwiązanie.

\section{I-węzły}

Przed rozszerzonymi atrybutami (rozdział \ref{sec:xattrs}) rozważane
było wykorzystanie i-węzłów do przechowywania limitów szybkości transferu danych.
